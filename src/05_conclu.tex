%%%%%%%%%%%%%%%%%%%%%%%%%%%%%%%%%%%%%%%%%%%%%%%%%%%%%%%%%%%%%%%%%%%%%%%%
% 5. おわりに
%%%%%%%%%%%%%%%%%%%%%%%%%%%%%%%%%%%%%%%%%%%%%%%%%%%%%%%%%%%%%%%%%%%%%%%%
\section{Conclusion}
In this paper, we proposed a system to visualize the number of people in an elevator and the degree of congestion on each floor to users. As future issues, we need to devise an alternative method for counting the number of people in an elevator using the number of BLE signals of COCOA, because the method cannot accurately count the number of people when the installation rate of COCOA is low, and it cannot capture Android terminals. In addition, we need to devise an algorithm to calculate the predicted waiting time based on the BLE signal data obtained in this study, and conduct an evaluation experiment to see if this system solves the congestion problem. In the future, we plan to integrate feedback mechanisms into our system to encourage behavioral changes, such as using the stairs instead of the elevator based on the IoT data-driven nudging concept \cite{nakamura2021iot}.
