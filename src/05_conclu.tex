%%%%%%%%%%%%%%%%%%%%%%%%%%%%%%%%%%%%%%%%%%%%%%%%%%%%%%%%%%%%%%%%%%%%%%%%
% 5. おわりに
%%%%%%%%%%%%%%%%%%%%%%%%%%%%%%%%%%%%%%%%%%%%%%%%%%%%%%%%%%%%%%%%%%%%%%%%
\section{Conclusion}
In this paper, we developed a system that measures the number of people in an elevator by measuring the COCOA signal, and visualizes in real time the congestion in the elevator and on each floor. We hypothesize that visualization of congestion to elevator users will encourage them to make a decision not to use the elevator. In the future, we plan to examine the long-term effect of this system on the increase or decrease of elevator utilization rate during congestion with and without this system. Furthermore we plan to integrate feedback mechanisms into our system to encourage behavioral changes, such as using the stairs instead of the elevator based on the IoT data-driven nudging concept \cite{nakamura2021iot}.
