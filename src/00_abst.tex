%%%%%%%%%%%%%%%%%%%%%%%%%%%%%%%%%%%%%%%%%%%%%%%%
% 概要
%%%%%%%%%%%%%%%%%%%%%%%%%%%%%%%%%%%%%%%%%%%%%%%%

\begin{abstract}
  % Due to the impact of the COVID-19, there has been a growing interest in visualizing the degree of congestion in public transportation and restaurants. Trains, buses, restaurants, and elevators are some of the places that tend to be crowded.

  % In this study, we developed a system to visualize the number of people in the elevator and the degree of congestion on each floor to the users to reduce the congestion in the elevator. The main purpose of this system is to present the congestion status in the elevator to the people who are waiting for the elevator, and to nudge them to make a decision such as to use the stairs or to delay the timing of pushing the elevator up/down button.

  % In some high-rise condominiums and large commercial facilities, cameras are installed in the elevators, and the images captured by the cameras are displayed on signage in the elevator halls. However, the camera-based method has some problems such as invasion of privacy and installation cost of the signage, and there are some places where large-scale construction cannot be done due to the structure of the building. Therefore, we propose a system that uses Bluetooth Low Energy (BLE) signals instead of cameras to measure the degree of congestion to ensure privacy, and that can be easily appended to most elevators.

  In this study, we propose a behavior change support system that encourages people to use stairs instead of elevators. Using the stairs is not only good for one's health, but nowadays it also plays a role in dense avoidance from the perspective of COVID-19.

  Although many people are waiting for the elevator, we thought that if we knew how many people were on the elevator, more people could change their behavior. However, there are few buildings where the number of people in the elevator is displayed on each floor. So, we have developed a headcount measurement system in the cargo and a visualization system, those can be easily appended to the current elevator.

  As a method to measure the number of people in an elevator, we propose a method to detect BLE signals transmitted from the terminals of elevator users who have installed COCOA, an application for confirming contact with the COVID-19 in Japan, and to measure in real time the number of detected BLE signals with a received signal strength (RSSI) exceeding a certain value. We propose a method to measure the number of BLE signals detected in real time.

  In this paper, we have designed and developed the system, verified the accuracy of the detection of the continuous operation time and the number of passengers, and estimated the behavior pattern and waiting time of elevator users using the system. The evaluation of the transformation of the decision making of the elevator users brought about by this system is out of the scope of this paper and will be discussed in the future.

  \keywords{Elevator \and Signage \and Nudge \and BLE \and COCOA}
\end{abstract}
