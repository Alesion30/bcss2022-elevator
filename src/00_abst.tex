%%%%%%%%%%%%%%%%%%%%%%%%%%%%%%%%%%%%%%%%%%%%%%%%
% 概要
%%%%%%%%%%%%%%%%%%%%%%%%%%%%%%%%%%%%%%%%%%%%%%%%

\begin{abstract}
  In this study, we propose a behavior change support system that encourages people to use stairs instead of elevators. Using the stairs is not only good for one's health, but nowadays it also plays a role in dense avoidance from the perspective of COVID-19.

  Although many people are waiting for the elevator, we thought that if we knew how many people were on the elevator, more people could change their behavior. However, there are few buildings where the number of people in the elevator is displayed on each floor. So, we have developed a headcount measurement system in the cargo and a visualization system, those can be easily appended to the current elevator.

  As a method to measure the number of people in an elevator, we propose a method to detect BLE (Bluetooth Low Energy) signals transmitted from the terminals of elevator users who have installed COCOA, an application for confirming contact with the COVID-19 in Japan, and to measure in real time the number of detected BLE signals with a received signal strength (RSSI) exceeding a certain value. We propose a method to measure the number of BLE signals detected in real time.

  In this paper, we have designed and developed the system, verified the accuracy of the detection of the continuous operation time and the number of passengers, and estimated the behavior pattern and waiting time of elevator users using the system. The evaluation of the transformation of the decision making of the elevator users brought about by this system is out of the scope of this paper and will be discussed in the future.

  \keywords{Elevator \and Signage \and Nudge \and BLE \and COCOA}
\end{abstract}
